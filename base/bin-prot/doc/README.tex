\documentclass[12pt]{article}

\usepackage{hevea}

%BEGIN LATEX
\usepackage{natbib}
%END LATEX

\newcommand{\mail}{\mailto{opensource@janestreet.com}}
\newcommand{\homeurl}{http://www.janestreet.com}
\newcommand{\athome}[2]{\ahref{\homeurl/#1}{#2}}
\newcommand{\www}{\athome{}{Markus Mottl}}

\newcommand{\ocaml}{\ahref{http://www.ocaml.org}{OCaml}}

\newcommand{\myhref}[1]{\ahref{#1}{#1}}
\newcommand{\ocsrc}[2]{\athome{ocaml/#1}{#2}}
\newcommand{\myocsrc}[1]{\athome{ocaml/#1}{#1}}

\newcommand{\janeshort}{\ahref{http://www.janestreet.com} {Jane Street Holding, LLC}}

\newcommand{\trow}[2]{\quad #1 \quad&\quad #2 \quad\\}
\newcommand{\trowl}[2]{\trow{#1}{#2}\hline}

%BEGIN LATEX
\newcommand{\theyear}{\number\year}
%END LATEX

\title{README: library ``Bin\_prot''}
\author{
  Copyright \quad (C) \quad \theyear \quad \janeshort \quad\\
  Author: Markus Mottl
}
\date{New York, 2007-10-12}

% DOCUMENT
\begin{document}
\maketitle
\section{Directory contents}
\begin{center}
\begin{tabular}{|c|c|}
\hline
\trowl{CHANGES}{History of code changes}
\trowl{COPYRIGHT}{Notes on copyright}
\trow{INSTALL}{Short notes on compiling and}
\trowl{}{installing the library}
\trowl{LICENSE}{``GNU LESSER GENERAL PUBLIC LICENSE''}
\trowl{LICENSE.Tywith}{License of Tywith, from which Bin-prot is derived}
\trowl{Makefile}{Top Makefile}
\trowl{OCamlMakefile}{Generic Makefile for OCaml-projects}
\trowl{OMakefile}{Ignore this file}
\trowl{README.txt}{This file}
\trowl{lib/}{OCaml-library for type-safe binary protocol conversions}
\trowl{lib\_test/}{Test applications for the Bin\_prot-library}
\end{tabular}
\end{center}

\section{What is ``Bin\_prot''}

This library contains functionality for reading and writing OCaml-values
in a type-safe binary protocol.  These functions are extremely efficient
and provide users with a convenient and safe way of performing I/O on any
extensionally defined data type.  This means that functions, objects,
and values whose type is bound through a polymorphic record field are
not supported, but everything else is.\\

As of now, there is no support for cyclic or shared values.
Cyclic values will lead to non-termination whereas shared values,
besides requiring significantly more space when encoded, may lead to
a substantial increase in memory footprint when they are read back in.

Currently only little endian\footnote{Endianness defines the byte
order in which machine representations for integers and floating point
numbers are written to main memory} computer architectures are supported.
Some architectures may potentially also suffer from data alignment issues
with this library.  Only Intel architectures are currently well-tested.
Both 32bit and 64bit architectures are fully supported.

\section{How can you use it?}

The API (.mli-files) in the library directory is fully documented.
Module \verb=Common= defines some globally used types, functions, exceptions,
and values.  \verb=Nat0= implements natural numbers including zero.\\

Modules \verb=Read_ml= and \verb=Write_ml= contain read and write functions
respectively for all basic types and are implemented in OCaml as
far as reasonable.  If you only want to read or write single, basic,
unstructured values, this module is probably the most efficient and
convenient for doing this.\\

Otherwise you should annotate your type definitions to generate
type converters automatically (see below).  The preprocessor in
\verb=pa_bin_prot.ml= will then generate highly optimized functions for
converting your OCaml-values to and from the binary representation.  This
automatically generated code will use functions in \verb=unsafe_common=,
\verb=unsafe_read_c= and \verb=unsafe_write_c=, which handle the basic
types with very low-level representations for efficiency.\\

The module \verb=Size= allows you to compute the size of basic
OCaml-values in the binary representations before writing them to
a buffer.  The code generator will also provide you with functions for
your user-defined types.\\

The modules \verb=Read_c= and \verb=Write_c= wrap the low-level converters
for basic values to ones accessible easily in OCaml and vice versa, and
export functions for wrapping user-defined converters.  This should make
it easy to add user-defined converters that interact with a specific
representation, but you want to make them available to the other one
quickly.  The test applications in the distribution make use of these
wrappers to verify the correctness of implementations for low-level (C)
and high-level (OCaml) representations.\\

The module \verb=Type_class= contains some extra definitions for type
classes of basic values.  These definitions can be passed to the function
\verb=bin_dump= in module \verb=Utils= to marshal values into buffers
of exact size using the binary protocol.  However, if bounds on the size
of values are known, it is usually more efficient to write them directly
into preallocated buffers and just catch exceptions if the buffer limits
are hit.  That way one does not have to compute the size of the value,
which can sometimes be almost as expensive as writing the value in the
first place.\\

The module \verb=Utils.ReadBuf= can be used to very efficiently
read size-prefixed values as written by \verb=bin_dump= with the
\verb=header_size= flag.  This works even in the presence of partial
data, e.g.\ when reading from streaming data coming from sockets, etc.
In most cases, when a whole value fits into a buffer, this module will
parse directly from the original buffer and only copy data to an internal
one on partial reads.

\subsection{Examples}

E.g. given the following type definition:

\begin{verbatim}
  type t = A | B
  with bin_io
\end{verbatim}

The above will generate the functions \verb=bin_size_t=,
\verb=bin_write_t=, \verb=bin_read_t=, and the type class values
\verb=bin_writer_t=, \verb=bin_reader_t= and \verb=bin_rw_t= .  If you
use the annotation \verb=bin_write= instead of \verb=bin_io=, then only
the write and size functions and their type class will be generated.
Specifying \verb=bin_read= will generate the read functions and associated
type class only.  \verb=bin_type_class= will generate the combined type
class only, thus allowing the user to easily define their own reader
and writer type classes.  The code generator may also generate low-level
entry points used for efficiency or backtracking.\\
\\
The preprocessor can also generate signatures for conversion functions.
Just add the wanted annotation to the type in a module signature for
that purpose.

\section{Specification of the Binary Protocol}

The binary protocol does not contain any data other than the minimum
needed to decode written out values.  This means that the user is
responsible for e.g.\ writing out the size of messages themselves if
they want to be able to preallocate sufficiently sized buffers before
reading.\\

The basic OCaml-values are written out character-wise as described below
using hex codes for the character encoding.  Some of these values require
size/length information to be written out before the value (e.g.\ for
lists, hash tables, strings, etc.).  Size information is always encoded
as natural numbers (\verb=Nat0.t=).\\

\noindent The following definitions will be used in the encoding
specifications below:

\begin{itemize}
\item \verb=CODE_NEG_INT8= $\rightarrow$ \verb=0xff=
\item \verb=CODE_INT16= $\rightarrow$ \verb=0xfe=
\item \verb=CODE_INT32= $\rightarrow$ \verb=0xfd=
\item \verb=CODE_INT64= $\rightarrow$ \verb=0xfc=
\end{itemize}

\subsection{Nat0.t}

If the value is:

\begin{itemize}
\item $<$ \verb=0x000000080= $\rightarrow$ lower 8 bit of the integer.
\item $<$ \verb=0x000010000= $\rightarrow$ \verb=CODE_INT16= followed by lower 16 bits of integer.
\item $<$ \verb=0x100000000= $\rightarrow$ \verb=CODE_INT32= followed by lower 32 bits of integer.
\item $\geq$ \verb=0x100000000= $\rightarrow$ \verb=CODE_INT64= followed by all 64 bits of integer\footnote{Only supported on 64 bit platforms.}.
\end{itemize}

Appropriate exceptions will be raised if there is an overflow, e.g.\ if
a 64 bit encoding is read on a 32 bit platform, or if the 32 bit or 64
bit encoding overflowed the 30 bit or 62 bit capacity\footnote{One
bit is reserved by OCaml for GC-tagging, and the sign bit is lost.}
of natural numbers on their respective platforms.

\subsection{Unit values}

\begin{itemize}
\item \verb=()= $\rightarrow$ \verb=0x00=
\end{itemize}

\subsection{Booleans}

\begin{itemize}
\item \verb=false= $\rightarrow$ \verb=0x00=
\item \verb=true= $\rightarrow$ \verb=0x01=
\end{itemize}

\subsection{Strings}

First the length of the string is written out as a \verb=Nat0.t=.
Then the contents of the string is copied verbatim.

\subsection{Characters}

Characters are written out verbatim.

\subsection{Integers}

This includes all integer types: \verb=int, int32, int64, nativeint=.
If the value is positive (including zero) and if it is:

\begin{itemize}
\item $<$ \verb=0x00000080= $\rightarrow$ lower 8 bit of the integer.
\item $<$ \verb=0x00008000= $\rightarrow$ \verb=CODE_INT16= followed by lower 16 bits of integer.
\item $<$ \verb=0x80000000= $\rightarrow$ \verb=CODE_INT32= followed by lower 32 bits of integer.
\item $\geq$ \verb=0x80000000= $\rightarrow$ \verb=CODE_INT64= followed by all 64 bits of integer.
\end{itemize}

\noindent If the value is negative and if it is:

\begin{itemize}
\item $\geq$ \verb=-0x00000080= $\rightarrow$ \verb=CODE_NEG_INT8= followed by lower 8 bit of integer.
\item $\geq$ \verb=-0x00008000= $\rightarrow$ \verb=CODE_INT16= followed by lower 16 bits of integer.
\item $\geq$ \verb=-0x80000000= $\rightarrow$ \verb=CODE_INT32= followed by lower 32 bits of integer.
\item $<$ \verb=-0x80000000= $\rightarrow$ \verb=CODE_INT64= followed by all 64 bits of integer.
\end{itemize}

All of the above branches will be considered when converting values
of type \verb=int64=.  The case for \verb=CODE_INT64= will only be
considered with types \verb=int= and \verb=nativeint= if the architecture
supports it.  \verb=int32= will never be encoded as a \verb=CODE_INT64=.
Appropriate exceptions will be raised if the architecture of or the type
requested by the reader does not support some encoding, or if there
is an overflow\footnote{An overflow can only happen with int values:
one bit is reserved by OCaml for the GC-tag.}.

\subsection{Floats}

Floats are written out according to the 64 bit IEEE 754 floating point
standard, i.e.\ their memory representation is copied verbatim.

\subsection{References and lazy values}

Same as the binary encoding of the value in the reference or of the
value calculated lazily.

\subsection{Option values}

If the value is:

\begin{itemize}
\item \verb=None= $\rightarrow$ \verb=0x00=
\item \verb=Some v= $\rightarrow$ \verb=0x01= followed by the encoding of \verb=v=.
\end{itemize}

\subsection{Tuples and records}

Values in tuples and records are written out one after the other in
the order as specified in the type specification without any extra
information.  Polymorphic record fields are supported unless a value
of the type bound by the field were accessed, which would lead to an
exception.

\subsection{Sum types}

Each tag is assigned an integer from $0$ to $n - 1$ in exactly the same
order as they occur in the type, where $n$ is the number of sum tags in
the type.  If a value of this type needs to be written out, then if:

\begin{itemize}
\item $n \leq 256 \rightarrow$ the integer associated with the tag is written out as one character (lower 8 bits).
\item $n \leq 65536 \rightarrow$ the integer associated with the tag is written out as two characters (lower 16 bits).
\end{itemize}

Arguments to the tag are written out in the order of occurrence without
any extra information.

\subsection{Polymorphic variants}

The tags of these values are written out as four characters, more
precisely as the 32 bit hash value computed by OCaml for the given tag.
Any arguments associated with the tag are written out afterwards in the
order of occurrence without any extra information.  When polymorphic
variants are being read, they will be matched in order of occurrence
(left-to-right) in the type and depth-first in the case of included
polymorphic types.  The first type containing a match for the variant
will be used for reading.

\subsection{Lists and arrays}

For lists and arrays the length is written out as a \verb=Nat0.t=
first, followed by all values in the same order as in the datastructure
without any extra information.

\subsection{Hash tables}

First the size of the hash table is written out as \verb=Nat0.t=.
Then the writer iterates over each binding in the hash table and writes
out the key followed by the value without any extra information.

\subsection{Bigarrays}

First the dimension(s) are written out as \verb=Nat0.t=.  Then the
contents is copied verbatim.

\subsection{Polymorphic values}

There is nothing special about polymorphic values as long as there are
conversion functions for the type parameters.  E.g.:

\begin{verbatim}
  type 'a t = A | B of 'a with bin_io
  type foo = int t with bin_io
\end{verbatim}

In the above case the conversion functions will behave as if \verb=foo=
had been defined as a monomorphic version of \verb=t= with \verb='a=
replaced by \verb=int= on the right hand side.

\subsection{Abstract datatypes}

If you want to convert an abstract datatype, you will have to roll your
own conversion functions.  Use the functions in module \verb=Read_c= and
\verb=Write_c= to map between low-level and high-level representations,
or implement those manually for maximum efficiency.

\section{Contact information}

\noindent In the case of bugs, feature requests and similar, you can
contact us here:\\\\

\hspace{2ex}\mail\\\\

\noindent Up-to-date information concerning this library should be
available here:\\\\

\hspace{2ex}\homeurl/ocaml\\\\

Enjoy!!\\\\

\end{document}
